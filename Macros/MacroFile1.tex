\newcommand{\PdfPsText}[2]{
  \ifpdf
     #1
  \else
     #2
  \fi
}

\newcommand{\IncludeGraphicsH}[3]{
  \PdfPsText{\includegraphics[height=#2]{#1}}{\includegraphics[bb = #3, height=#2]{#1}}
}

\newcommand{\IncludeGraphicsW}[3]{
  \PdfPsText{\includegraphics[width=#2]{#1}}{\includegraphics[bb = #3, width=#2]{#1}}
}

\newcommand{\InsertFig}[3]{
  \begin{figure}[!htbp]
    \begin{center}
      \leavevmode
      #1
      \caption{#2}
      \label{#3}
    \end{center}
  \end{figure}
}

\newcommand{\mc}[3]{\multicolumn{#1}{#2}{#3}}
\newcommand{\mr}[3]{\multirow{#1}{#2}{#3}}

\newcommand{\authornames}{Alfredo Kalaitzis}
\newcommand{\thesistitle}{Learning with structured covariance matrices in linear Gaussian models}

\newcommand\Declaration[1]{
%   \btypeout{Declaration of Authorship}
%   \addtotoc{Declaration of Authorship}
  \thispagestyle{plain}
  \null\vfil
  %\vskip 60\p@
  \begin{center}{
    \huge\bf Declaration of Authorship \par}
  \end{center}
  %\vskip 60\p@
  {\normalsize #1}
  \vfil\vfil\null
  %\cleardoublepage
}

\newcommand{\mb}[1]{ \mathbf{#1} }

\newcommand{\bphi}{ \boldsymbol{\phi} }
\newcommand{\bepsilon}{ \boldsymbol{\epsilon} }
\newcommand{\btheta}{ \boldsymbol{\theta} }
\newcommand{\bmu}{ \boldsymbol{\mu} }
\newcommand{\bSigma}{ \mb{\Sigma} }
\newcommand{\bPhi}{ \mb{\Phi} }
\newcommand{\balpha}{ \boldsymbol{\alpha} }
\newcommand{\bLambda}{ \mb{\Lambda} }
\newcommand{\bpi}{ \boldsymbol{\pi} }
\newcommand{\bGamma}{ \mb{\Gamma} }
\newcommand{\bPsi}{ \mb{\Psi} }
\newcommand{\bpsi}{ \boldsymbol{\psi} }
\newcommand{\bTheta}{ \mb{\Theta} }
\newcommand{\bOmega}{ \mb{\Omega} }


\newcommand{\Realspace}{ \mathbb{R} }
\newcommand{\Normal}[2]{ \mathcal{N} \left( #1, #2 \right) }

\newcommand{\E}[1]{ \mathbb{E}\left[#1\right] }
\newcommand{\V}[1]{ \textrm{var}\left[#1\right] }
\newcommand{\C}[2]{ \textrm{cov}\left[#1,#2\right] }
\newcommand{\verts}[1]{ \lvert #1 \rvert }
\newcommand{\tr}[1]{ \textrm{tr} \left( #1 \right) }
\newcommand{\inner}[2]{ \langle #1,#2 \rangle }
\newcommand{\norm}[1]{ \vert\vert #1 \vert\vert }
\newcommand{\mat}[1]{ \begin{bmatrix} #1 \end{bmatrix} }
\newcommand{\deriv}[1]{ \tfrac{\textrm{d}}{\textrm{d} #1} }
\newcommand{\pderiv}[1]{ \tfrac{\partial}{\partial #1} }
\newcommand{\expo}[1]{\textrm{exp}\left\{ #1 \right\}}
\newcommand{\diag}[1]{\textrm{diag}\left( #1 \right)}
\newcommand{\ceq}{\overset{c}{=}}
\newcommand{\sm}{\setminus}

%%% Local Variables: 
%%% mode: latex
%%% TeX-master: "~/Documents/LaTeX/CUEDThesisPSnPDF/thesis"
%%% End: 
